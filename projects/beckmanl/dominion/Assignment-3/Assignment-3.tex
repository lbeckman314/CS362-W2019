% Created 2019-02-10 Sun 22:31
% Intended LaTeX compiler: pdflatex
\documentclass[11pt]{article}
\usepackage[utf8]{inputenc}
\usepackage[T1]{fontenc}
\usepackage{graphicx}
\usepackage{grffile}
\usepackage{longtable}
\usepackage{wrapfig}
\usepackage{rotating}
\usepackage[normalem]{ulem}
\usepackage{amsmath}
\usepackage{textcomp}
\usepackage{amssymb}
\usepackage{capt-of}
\usepackage{hyperref}
\input{/home/liam/main.tex}
\author{liam beckman}
\date{}
\title{Assignment 3}
\hypersetup{
 pdfauthor={liam beckman},
 pdftitle={Assignment 3},
 pdfkeywords={},
 pdfsubject={},
 pdfcreator={Emacs 26.1 (Org mode 9.1.9)}, 
 pdflang={English}}
\begin{document}

\maketitle

\section{Bugs}
\label{sec:bugs}

A prominent bug is one I introduced in Assignment 2 within the \texttt{myAdventurer} function.

\begin{listing}[htbp]
\begin{minted}[breaklines=true,breakanywhere=true]{c}
int myAdventurer(int drawntreasure, int currentPlayer, int temphand[], struct gameState *state)
{
    // ...

    // BUG: gold is not counted toward drawntreasure
    //if (cardDrawn == copper || cardDrawn == silver || cardDrawn == gold)
    if (cardDrawn == copper || cardDrawn == silver)
        drawntreasure++;
    
    // ...
}
\end{minted}
\caption{myAdventurer Bug that doesn't count gold toward treasure.}
\end{listing}

\section{Unit Testing}
\label{sec:unittesting}

\subsection{Unit Tests}
\label{sec:unittests}

The four unit tests covered the following four functions:
\begin{enumerate}
\item getCost
\item isGameOver
\item compare
\item updateCoins
\end{enumerate}

\begin{table}[htbp]
\caption{Coverage for Function Tests}
\centering
\begin{tabular}{lll}
function & statement coverage & branch coverage\\
\hline
getCost & 1.95\% & 6.75\%\\
isGameOver & 1.62\% & 1.93\%\\
compare & 0.97\% & 0.96\%\\
updateCoins & 2.27\% & 1.93\%\\
\end{tabular}
\end{table}

\subsubsection{unittest1 (getCost function)}
\label{sec:unittestgetcostfunction}

\begin{verbatim}
Lines executed:1.95% of 616
Branches executed:6.75% of 415
Taken at least once:1.20% of 415
Calls executed:0.00% of 108
\end{verbatim}

\subsubsection{unittest2 (isGameOver function)}
\label{sec:unittestisgameoverfunction}

\begin{verbatim}
Lines executed:1.62% of 616
Branches executed:1.93% of 415
Taken at least once:1.93% of 415
Calls executed:0.00% of 108
\end{verbatim}

\subsubsection{unittest3 (compare function)}
\label{sec:unittestcomparefunction}

\begin{verbatim}
Lines executed:0.97% of 616
Branches executed:0.96% of 415
Taken at least once:0.96% of 415
Calls executed:0.00% of 108
\end{verbatim}

\subsubsection{unittest4 (updateCoins function)}
\label{sec:unittestupdatecoinsfunction}

\begin{verbatim}
Lines executed:2.27% of 616
Branches executed:1.93% of 415
Taken at least once:1.69% of 415
Calls executed:0.00% of 108
\end{verbatim}

\subsection{Card Tests}
\label{sec:cardtests}

The four card tests covered the following four card types:
\begin{enumerate}
\item smithy
\item adventurer
\item village
\item mine
\end{enumerate}

\begin{table}[htbp]
\caption{Coverage for Card Tests}
\centering
\begin{tabular}{lll}
card & statement coverage & branch coverage\\
\hline
Smithy & 6.49\% & 3.86\%\\
Adventurer & 17.05\% & 17.83\%\\
Village & 19.32\% & 22.89\%\\
Mine & 21.27\% & 28.92\%\\
\end{tabular}
\end{table}

\subsubsection{cardtest1 (smithy)}
\label{sec:cardtestsmithy}

\begin{verbatim}
Lines executed:6.49% of 616
Branches executed:3.86% of 415
Taken at least once:2.41% of 415
Calls executed:2.78% of 108
\end{verbatim}

\subsubsection{cardtest2 (adventurer)}
\label{sec:cardtestadventurer}

\begin{verbatim}
Lines executed:17.05% of 616
Branches executed:17.83% of 415
Taken at least once:13.98% of 415
Calls executed:10.19% of 108
\end{verbatim}

\subsubsection{cardtest3 (village)}
\label{sec:cardtestvillage}

\begin{verbatim}
Lines executed:19.32% of 616
Branches executed:22.89% of 415
Taken at least once:13.98% of 415
Calls executed:11.11% of 108
\end{verbatim}

\subsubsection{cardtest4 (mine)}
\label{sec:cardtestmine}

\begin{verbatim}
Lines executed:21.27% of 616
Branches executed:28.92% of 415
Taken at least once:16.63% of 415
Calls executed:12.04% of 108
\end{verbatim}

\section{Unit Testing Efforts}
\label{sec:unittestingefforts}

\subsection{Function Testing}
\label{sec:functiontesting}

Funciton testing involved confirming that a given function had the desired effect. For example, testing the \texttt{isGameOver} function involved first confirming that a given game is still continuing, then changing a given value to trigger an end to the game (e.g. changing three supply counts to 0), and finally testing for a successful end of game value:

\begin{listing}[htbp]
\begin{minted}[breaklines=true,breakanywhere=true]{c}
myAssert(isGameOver(&testGame) == 0);

for (i = 0; i < 3; i++)
{
    testGame.supplyCount[i] = 0;
}

myAssert(isGameOver(&testGame) == 1);
\end{minted}
\caption{Function testing example.}
\end{listing}

\subsection{Card Testing}
\label{sec:cardtesting}

Testing cards involved two steps:
\begin{enumerate}
\item Confirming that the implementation exited correctly by asserting that the relevant function returned 0 (as opposed to -1 for an unsuccessful exit).
\item Confirming that the actions of the card had the desired effect. For the following smithy example, this means that if a player started with five cards in both their hands and decks, then the smithy card would result in a deck count of two and a hand count of seven.
\end{enumerate}

\begin{listing}[htbp]
\begin{minted}[breaklines=true,breakanywhere=true]{c}
// initialize hand and deck counts
testGame->deckCount[player] = 5;
testGame->handCount[player] = 5;

// check successful exit of function
myAssert(mySmithy(player, testGame, handPos) == 0); 

// check successful 
myAssert(testGame->deckCount[player] == 2);
myAssert(testGame->handCount[player] == 7);
\end{minted}
\caption{Card testing example.}
\end{listing}

As per the instructor recommendation all of the unit tests and the card tests utilized a custom assert implementation, so as to be able to collect coverage information even when a crash fails (this behavior is not included in the standard C \texttt{assert} function).

\begin{listing}[htbp]
\begin{minted}[breaklines=true,breakanywhere=true]{c}
#define myAssert(expression)                          \
    if (expression)                                   \
        (passed)(#expression, __LINE__, __FILE__, 0); \
    else                                              \
        (failed)(#expression, __LINE__, __FILE__, 0); \

// outputs "TEST FAILED" message
void failed(char *expression, int line, char *file, int color);

// outputs "TEST PASSED" message
void passed(char *expression, int line, char *file, int color);
\end{minted}
\caption{Custom assert macro for use in the test suite.}
\end{listing}
\end{document}