% Created 2019-03-16 Sat 23:54
% Intended LaTeX compiler: pdflatex
\documentclass[11pt]{article}
\usepackage[utf8]{inputenc}
\usepackage[T1]{fontenc}
\usepackage{graphicx}
\usepackage{grffile}
\usepackage{longtable}
\usepackage{wrapfig}
\usepackage{rotating}
\usepackage[normalem]{ulem}
\usepackage{amsmath}
\usepackage{textcomp}
\usepackage{amssymb}
\usepackage{capt-of}
\usepackage{hyperref}
\input{/home/liam/main.tex}
\author{liam beckman}
\date{\today}
\title{Assignment 5}
\hypersetup{
 pdfauthor={liam beckman},
 pdftitle={Assignment 5},
 pdfkeywords={},
 pdfsubject={},
 pdfcreator={Emacs 26.1 (Org mode 9.1.9)}, 
 pdflang={English}}
\begin{document}

\maketitle


\section{Bug-Reports}
\label{sec:bugreports}
\subsection{Bug 1}
\label{sec:bug}

\begin{itemize}
\item Title: Wrong Smithy draw card value.
\item Project: dominion
\item Reporter: liam beckman
\item Date: 16 March 2019
\item Type: Bug
\item File Name: cardtest1.c
\item Environment: NVIM v0.3.4 (Build type: Release)
\item Description:
\end{itemize}

The Smithy card (implemented in the \texttt{play\_smithy} function) drew a total of four cards, instead of the correct amount of three cards. This is likely due to the incorrect termination expression of \texttt{i < 4} in the primary for loop.

\begin{minted}[linenos,firstnumber=1,breaklines=true,breakanywhere=true]{c}
int play_smithy(int currentPlayer, int handPos, struct gameState *state)
{
        //+3 Cards
      for (int i = 0; i < 4; i++)
    {
      drawCard(currentPlayer, state);
    }
            
      //discard card from hand
      discardCard(handPos, currentPlayer, state, 0);
      return 0;
}
\end{minted}

\subsection{Bug 2}
\label{sec:bug1}

\begin{itemize}
\item Title: Adventurer Card Error
\item Project: dominion
\item Reporter: liam beckman
\item Date: 16 March 2019
\item Type: Bug
\item File Name: cardtest2.c
\item Environment: NVIM v0.3.4 (Build type: Release)
\item Description:

Uninitialized variable \texttt{z} in \texttt{play\_adventurer} function within the \texttt{temphand[z]=cardDrawn} statement. Furthermore, the \texttt{z} variable is not incremented after the \texttt{temphand[z]=cardDrawn} statement, so the first \texttt{temphand} card will continually be overwritten every iteration of the \texttt{while(drawntreasure<2)} statement.

\begin{minted}[linenos,firstnumber=1,breaklines=true,breakanywhere=true]{c}
while(drawntreasure<2)

//...

    if (cardDrawn == copper || cardDrawn == silver || cardDrawn == gold)
        drawntreasure++;
    else
    {
        temphand[z]=cardDrawn;
        state->
    handCount[currentPlayer]--; //this should just remove the top card (the most recently drawn one).
    }


\end{minted}
\end{itemize}

\section{Test-Report}
\label{sec:testreport}

I chose to use Yan Ming's branch for their second assignment (yanme-assignment-2), as her later branches were such that no prominent bugs could be found. Thus, I am confident in saying that their code is reliable and well executed, as I was forced to analyze code with purposeful bugs implemented.

\subsection{Manual Review}
\label{sec:manualreview}
\subsubsection{cardEffect}
\label{sec:cardeffect}

Missing end bracket in \texttt{cardEffect} function.

\begin{minted}[linenos,firstnumber=1,breaklines=true,breakanywhere=true]{c}
int cardEffect(int card, int choice1, int choice2, int choice3, struct gameState *state, int handPos, int *bonus)

// ...
\end{minted}


Variable misspelling of \texttt{drawnteasure} variable in \texttt{play\_adventurer} function. Initialized as \texttt{drawnteasure}.

\subsubsection{play\_adventurer}
\label{sec:playadventurer}

\begin{minted}[linenos,firstnumber=1,breaklines=true,breakanywhere=true]{c}
int play_adventurer(int currentPlayer, int temphand[MAX_HAND],  struct gameState *state)


int drawnteasure = 0;
int cardDrawn;
while(drawntreasure<2)

    // ...
\end{minted}

Uninitialized variable \texttt{z} in \texttt{play\_adventurer} function within the \texttt{temphand[z]=cardDrawn} statement. Furthermore, the \texttt{z} variable is not incremented after the \texttt{temphand[z]=cardDrawn} statement, so the first \texttt{temphand} card will continually be overwritten every iteration of the \texttt{while(drawntreasure<2)} statement.


\begin{minted}[linenos,firstnumber=1,breaklines=true,breakanywhere=true]{c}
while(drawntreasure<2)

//...

    if (cardDrawn == copper || cardDrawn == silver || cardDrawn == gold)
        drawntreasure++;
    else
    {
        temphand[z]=cardDrawn;
        state->
    handCount[currentPlayer]--; //this should just remove the top card (the most recently drawn one).
    }


\end{minted}

\subsubsection{play\_smithy}
\label{sec:playsmithy}

Uninitialized variable \texttt{i} in \texttt{play\_smithy} function.

\begin{minted}[linenos,firstnumber=1,breaklines=true,breakanywhere=true]{c}
int play_smithy(int currentPlayer, int handPos, struct gameState *state)

//+3 Cards
    for (i = 0; i < 4; i++)

        // ...
\end{minted}

\subsubsection{play\_mine}
\label{sec:playmine}

Uninitialized variables \texttt{j} and \texttt{i} in \texttt{play\_mine} function.

\begin{minted}[linenos,firstnumber=1,breaklines=true,breakanywhere=true]{c}
int play_mine ( int currentPlayer, int choice1, int choice2, struct gameState *state)

    j = state->
    hand[currentPlayer][choice1];  //store card we will trash

// ...

//discard trashed card
for (i = 0; i < state->
    handCount[currentPlayer]; i++)
{
    if (state->
    hand[currentPlayer][i] == j)
    {
        discardCard(i, currentPlayer, state, 0);
        break;
    }
}

return 0;

\end{minted}

Undeclared variable \texttt{handPos} in \texttt{play\_mine} function.

\begin{minted}[linenos,firstnumber=1,breaklines=true,breakanywhere=true]{c}
int play_mine ( int currentPlayer, int choice1, int choice2, struct gameState *state)

// ...

//discard card from hand
    discardCard(handPos, currentPlayer, state, 0);
\end{minted}

\subsubsection{play\_council}
\label{sec:playcouncil}

Uninitialized variable \texttt{i} in \texttt{play\_council} function.

\begin{minted}[linenos,firstnumber=1,breaklines=true,breakanywhere=true]{c}
int play_council(int currentPlayer, int handPos, struct gameState *state)
{
    //+4 Cards
    for (i = 0; i < 4; i++)

        // ...
\end{minted}


\subsection{Unit Testing}
\label{sec:unittesting}

\subsubsection{Test Results}
\label{sec:testresults}

\begin{enumerate}
\item getCost Function
\label{sec:getcostfunction}
\begin{verbatim}
 1  unittest1.c:17 TEST SUCCESSFULLY COMPLETED ->
 2       getCost(0) == 0
 3  unittest1.c:20 TEST SUCCESSFULLY COMPLETED ->
 4       getCost(1) == 2
 5  unittest1.c:23 TEST SUCCESSFULLY COMPLETED ->
 6       getCost(2) == 5
 7  unittest1.c:26 TEST SUCCESSFULLY COMPLETED ->
 8       getCost(3) == 8
 9  unittest1.c:29 TEST SUCCESSFULLY COMPLETED ->
10       getCost(4) == 0
\end{verbatim}

\item isGameOver Function
\label{sec:isgameoverfunction}
\begin{verbatim}
1  unittest2.c:28 TEST SUCCESSFULLY COMPLETED ->
2       isGameOver(&testGame) == 0
3  unittest2.c:35 TEST SUCCESSFULLY COMPLETED ->
4       isGameOver(&testGame) == 1
5  unittest2.c:43 TEST SUCCESSFULLY COMPLETED ->
6       isGameOver(&testGame) == 0
7  unittest2.c:46 TEST SUCCESSFULLY COMPLETED ->
8       isGameOver(&testGame) == 1
\end{verbatim}

\item compare Function
\label{sec:comparefunction}
\begin{verbatim}
1  unittest3.c:14 TEST SUCCESSFULLY COMPLETED ->
2       compare(b, a) == 1
3  unittest3.c:15 TEST SUCCESSFULLY COMPLETED ->
4       compare(a, b) == -1
5  unittest3.c:16 TEST SUCCESSFULLY COMPLETED ->
6       compare(a, a) == 0
7  unittest3.c:17 TEST SUCCESSFULLY COMPLETED ->
8       compare(b, b) == 0
\end{verbatim}

\item updateCoins Function
\label{sec:updatecoinsfunction}
\begin{verbatim}
1  unittest4.c:31 TEST SUCCESSFULLY COMPLETED ->
2       testGame->coins == 9
3  unittest4.c:43 TEST SUCCESSFULLY COMPLETED ->
4       testGame->coins == 8
\end{verbatim}

\item Smithy Card
\label{sec:smithycard}

\begin{enumerate}
\item Failed Tests
\label{sec:failedtests}

\begin{verbatim}
 1  cardtest1.c:14 TEST SUCCESSFULLY COMPLETED ->
 2       play_smithy(player, handPos, testGame) == 0
 3  cardtest1.c:15 TEST FAILED: ->
 4       testGame->deckCount[player] == 2
 5  cardtest1.c:16 TEST FAILED: ->
 6       testGame->handCount[player] == 7
 7  cardtest1.c:19 TEST SUCCESSFULLY COMPLETED ->
 8       testGame->deckCount[player] == 0
 9  cardtest1.c:20 TEST SUCCESSFULLY COMPLETED ->
10       testGame->handCount[player] == 8
\end{verbatim}


\item Successful Tests
\label{sec:successfultests}

\begin{verbatim}
 1  cardtest1.c:14 TEST SUCCESSFULLY COMPLETED ->
 2       play_smithy(player, handPos, testGame) == 0
 3  cardtest1.c:15 TEST SUCCESSFULLY COMPLETED ->
 4       testGame->deckCount[player] == 2
 5  cardtest1.c:16 TEST SUCCESSFULLY COMPLETED ->
 6       testGame->handCount[player] == 7
 7  deck count: 2
 8  hand count: 7
 9  cardtest1.c:21 TEST SUCCESSFULLY COMPLETED ->
10       testGame->deckCount[player] == 0
11  cardtest1.c:22 TEST SUCCESSFULLY COMPLETED ->
12       testGame->handCount[player] == 8
13  File 'cardtest1.c'
14  Lines executed:100.00% of 14
15  Branches executed:100.00% of 10
16  Taken at least once:50.00% of 10
17  Calls executed:66.67% of 15
18  Creating 'cardtest1.c.gcov'
\end{verbatim}

\item Code Changed
\label{sec:codechanged}

\begin{minted}[linenos,firstnumber=1,breaklines=true,breakanywhere=true]{c}
int play_smithy(int currentPlayer, int handPos, struct gameState *state)
{
        //+3 Cards
        // changed "i < 4" to "i < 3"
      for (int i = 0; i < 3; i++)
    {
      drawCard(currentPlayer, state);
    }
            
      //discard card from hand
      discardCard(handPos, currentPlayer, state, 0);
      return 0;
}
\end{minted}
\end{enumerate}


\item Adventurer Card
\label{sec:adventurercard}
\begin{verbatim}
1  cardtest2.c:19 TEST SUCCESSFULLY COMPLETED ->
2       play_adventurer(player, temphand, testGame) == 0
3  cardtest2.c:22 TEST SUCCESSFULLY COMPLETED ->
4       lastCard == copper || lastCard == silver || lastCard == gold
\end{verbatim}

\item Village Card
\label{sec:villagecard}
\begin{verbatim}
1  cardtest3.c:31 TEST SUCCESSFULLY COMPLETED ->
2       cardEffect(card, choice1, choice2, choice3, testGame, handPos, bonus) == 0
3  cardtest3.c:35 TEST SUCCESSFULLY COMPLETED ->
4       actions == actionsOld + 2
\end{verbatim}

\item Mine Card
\label{sec:minecard}
\begin{verbatim}
1  cardtest4.c:21 TEST SUCCESSFULLY COMPLETED ->
2       play_mine(player, choice1, choice2, testGame, handPos) == 0
\end{verbatim}
\end{enumerate}

\subsubsection{Code Coverage}
\label{sec:codecoverage}

\begin{enumerate}
\item getCost Function
\label{sec:getcostfunction1}
\begin{verbatim}
 1  File 'unittest1.c'
 2  Lines executed:100.00% of 6
 3  Branches executed:100.00% of 10
 4  Taken at least once:50.00% of 10
 5  Calls executed:66.67% of 15
 6  Creating 'unittest1.c.gcov'
 7  
 8  File 'dominion.c'
 9  Lines executed:2.00% of 599
10  Branches executed:6.85% of 409
11  Taken at least once:1.22% of 409
12  Calls executed:0.00% of 92
13  Creating 'dominion.c.gcov'
\end{verbatim}

\item isGameOver Function
\label{sec:isgameoverfunction1}
\begin{verbatim}
 1  File 'unittest2.c'
 2  Lines executed:100.00% of 17
 3  Branches executed:100.00% of 12
 4  Taken at least once:66.67% of 12
 5  Calls executed:71.43% of 14
 6  Creating 'unittest2.c.gcov'
 7  
 8  File 'dominion.c'
 9  Lines executed:1.67% of 599
10  Branches executed:1.96% of 409
11  Taken at least once:1.96% of 409
12  Calls executed:0.00% of 92
13  Creating 'dominion.c.gcov'
\end{verbatim}

\item compare Function
\label{sec:comparefunction1}
\begin{verbatim}
 1  File 'unittest3.c'
 2  Lines executed:100.00% of 9
 3  Branches executed:100.00% of 8
 4  Taken at least once:50.00% of 8
 5  Calls executed:66.67% of 12
 6  Creating 'unittest3.c.gcov'
 7  
 8  File 'dominion.c'
 9  Lines executed:1.00% of 599
10  Branches executed:0.98% of 409
11  Taken at least once:0.98% of 409
12  Calls executed:0.00% of 92
13  Creating 'dominion.c.gcov'
\end{verbatim}

\item updateCoins Function
\label{sec:updatecoinsfunction1}
\begin{verbatim}
 1  File 'unittest4.c'
 2  Lines executed:100.00% of 25
 3  Branches executed:100.00% of 4
 4  Taken at least once:50.00% of 4
 5  Calls executed:77.78% of 9
 6  Creating 'unittest4.c.gcov'
 7  
 8  File 'dominion.c'
 9  Lines executed:2.34% of 599
10  Branches executed:1.96% of 409
11  Taken at least once:1.71% of 409
12  Calls executed:0.00% of 92
13  Creating 'dominion.c.gcov'
\end{verbatim}

\item Smithy Card
\label{sec:smithycard1}
\begin{verbatim}
1  File 'cardtest1.c'
2  Lines executed:100.00% of 12
3  Branches executed:100.00% of 10
4  Taken at least once:50.00% of 10
5  Calls executed:61.54% of 13
6  Creating 'cardtest1.c.gcov'
\end{verbatim}

\item Adventurer Card
\label{sec:adventurercard1}
\begin{verbatim}
1  File 'cardtest2.c'
2  Lines executed:100.00% of 13
3  Branches executed:50.00% of 8
4  Taken at least once:25.00% of 8
5  Calls executed:71.43% of 7
6  Creating 'cardtest2.c.gcov'
\end{verbatim}

\item Village Card
\label{sec:villagecard1}
\begin{verbatim}
1  File 'cardtest3.c'
2  Lines executed:100.00% of 19
3  Branches executed:100.00% of 4
4  Taken at least once:50.00% of 4
5  Calls executed:71.43% of 7
6  Creating 'cardtest3.c.gcov'
\end{verbatim}

\item Mine Card
\label{sec:minecard1}
\begin{verbatim}
1  File 'cardtest4.c'
2  Lines executed:100.00% of 11
3  Branches executed:100.00% of 2
4  Taken at least once:50.00% of 2
5  Calls executed:80.00% of 5
6  Creating 'cardtest4.c.gcov'
\end{verbatim}
\end{enumerate}

\subsection{Random Testing}
\label{sec:randomtesting}

\subsubsection{Test Results}
\label{sec:testresults1}

\begin{enumerate}
\item Smithy
\label{sec:smithy}
\begin{verbatim}
 1  numPlayers: 4
 2  thisPlayer: 3
 3  ----------------- Testing Card: smithy ----------------
 4  TEST 1: random test
 5  randomtestcard1.c:65 TEST SUCCESSFULLY COMPLETED ->
 6       testG.handCount[thisPlayer] == G.handCount[thisPlayer] + 2
 7  randomtestcard1.c:66 TEST SUCCESSFULLY COMPLETED ->
 8       testG.deckCount[thisPlayer] == G.deckCount[thisPlayer] - 3
 9  
10   >>>>> SUCCESS: Testing complete smithy <<<<<
\end{verbatim}

\item Mine
\label{sec:mine}
\begin{verbatim}
 1  ----------------- Testing Card: mine ----------------
 2  TEST 1: random test
 3  randomtestcard2.c:56 TEST SUCCESSFULLY COMPLETED ->
 4       getCost(testG.hand[thisPlayer][choice1]) + 3 <= getCost(choice2)
 5  randomtestcard2.c:65 TEST SUCCESSFULLY COMPLETED ->
 6       testG.handCount[thisPlayer] == G.handCount[thisPlayer]
 7  randomtestcard2.c:71 TEST SUCCESSFULLY COMPLETED ->
 8       testG.supplyCount[choice2] == G.supplyCount[choice2]
 9  
10   >>>>> SUCCESS: Testing complete mine <<<<<
\end{verbatim}

\item Adventurer
\label{sec:adventurer}
\begin{verbatim}
1  numplayers: 4
2  thisplayer: 3
3  ----------------- Testing Card: adventurer ----------------
4  TEST 1: random test
5  randomtestcardadventurer.c:88 TEST SUCCESSFULLY COMPLETED ->
6       drawntreasure == drawntreasurePre + 2
7  
8   >>>>> SUCCESS: Testing complete adventurer <<<<<
9      
\end{verbatim}
\end{enumerate}

\subsubsection{Code Coverage}
\label{sec:codecoverage1}

\begin{enumerate}
\item Smithy
\label{sec:smithy1}
\begin{verbatim}
 1  File 'randomtestcard1.c'
 2  Lines executed:100.00% of 23
 3  Branches executed:100.00% of 4
 4  Taken at least once:50.00% of 4
 5  Calls executed:88.24% of 17
 6  Creating 'randomtestcard1.c.gcov'
 7  
 8  File 'myAssert.c'
 9  Lines executed:35.71% of 14
10  Branches executed:50.00% of 8
11  Taken at least once:25.00% of 8
12  Calls executed:16.67% of 6
13  Creating 'myAssert.c.gcov'
\end{verbatim}

\item Mine
\label{sec:mine1}
\begin{verbatim}
 1  File 'randomtestcard2.c'
 2  Lines executed:100.00% of 23
 3  Branches executed:100.00% of 8
 4  Taken at least once:62.50% of 8
 5  Calls executed:86.36% of 22
 6  Creating 'randomtestcard2.c.gcov'
 7  
 8  File 'myAssert.c'
 9  Lines executed:71.43% of 14
10  Branches executed:100.00% of 8
11  Taken at least once:50.00% of 8
12  Calls executed:33.33% of 6
13  Creating 'myAssert.c.gcov'
14      
\end{verbatim}

\item Adventurer
\label{sec:adventurer1}
\begin{verbatim}
 1  File 'randomtestcardadventurer.c'
 2  Lines executed:81.82% of 33
 3  Branches executed:33.33% of 18
 4  Taken at least once:16.67% of 18
 5  Calls executed:92.31% of 13
 6  Creating 'randomtestcardadventurer.c.gcov'
 7  
 8  File 'myAssert.c'
 9  Lines executed:35.71% of 14
10  Branches executed:50.00% of 8
11  Taken at least once:25.00% of 8
12  Calls executed:16.67% of 6
13  Creating 'myAssert.c.gcov'
\end{verbatim}
\end{enumerate}


\section{Debugging}
\label{sec:debugging}
\subsection{gdb}
\label{sec:gdb}

The GNU Debugger was used to track down a bug in the Smithy card in dominion.c (implemented in the \texttt{play\_smithy} function). By creating a break point at the \texttt{play\_smithy} function, we were able to first confirm the initial values of the variables (hand count and deck count), and then step into the \texttt{play\_smithy} function. In this case the example was readily apparent, as the \texttt{for (int i = 0; i < 4; i++)} statement should be \texttt{for (int i = 0; i < 3; i++)} for the Smithy card to draw the correct amount of three cards (and not four as the previous version had it doing).

\begin{verbatim}
 1  $ gdb cardtest1
 2  GNU gdb (GDB) 8.2.1
 3  Copyright (C) 2018 Free Software Foundation, Inc.
 4  License GPLv3+: GNU GPL version 3 or later <http://gnu.org/licenses/gpl.html>                                                                         
 5  This is free software: you are free to change and redistribute it.
 6  There is NO WARRANTY, to the extent permitted by law.
 7  Type "show copying" and "show warranty" for details.
 8  This GDB was configured as "x86_64-pc-linux-gnu".
 9  Type "show configuration" for configuration details.
10  For bug reporting instructions, please see:
11  <http://www.gnu.org/software/gdb/bugs/>.
12  Find the GDB manual and other documentation resources online at:
13  For help, type "help".
14  Type "apropos word" to search for commands related to "word"...
15  Reading symbols from cardtest1...done.
16  (gdb) b 20                                                        
17  Breakpoint 1 at 0x2471: file cardtest1.c, line 20. 
18  (gdb) b 21                         
19  Breakpoint 2 at 0x2484: file cardtest1.c, line 21.
20  (gdb) r                            
21  Starting program: /home/liam/Documents/code/osu/2019winter/cs362-software/CS362-W2019/projects/yanmeDominion/projects/yanme/dominion/cardtest1        
22  cardtest1.c:14 TEST SUCCESSFULLY COMPLETED -> play_smithy(player, handPos, testGame) == 0                                                             
23  cardtest1.c:15 TEST SUCCESSFULLY COMPLETED -> testGame->deckCount[player] == 2                                                                        
24  cardtest1.c:16 TEST SUCCESSFULLY COMPLETED -> testGame->handCount[player] == 7                                                                        
25  deck count: 2                            
26  hand count: 7                      
27                                          
28  Breakpoint 1, main (argc=1, argv=0x7fffffffe0d8) at cardtest1.c:20
29  20          play_smithy(player, handPos, testGame);      
30  (gdb) p testGame->deckCount[player]
31  $1 = 2                        
32  (gdb) p testGame->handCount[player]
33  $2 = 7                                           
34  (gdb) s
35  play_smithy (currentPlayer=0, handPos=0, state=0x555555569260)
36      at dominion.c:681
37  681           for (int i = 0; i < 4; i++)
38  (gdb) s
39  683               drawCard(currentPlayer, state);
40  (gdb) c
41  Continuing.                          
42  Breakpoint 2, main (argc=1, argv=0x7fffffffe0d8) at cardtest1.c:21
43  21          myAssert(testGame->deckCount[player] == 0);
44  (gdb) p testGame->deckCount[player]                                                                                                                   
45  $3 = 0                                                            
46  (gdb) p testGame->handCount[player]                  
47  $4 = 8                                              
48  (gdb) q                                          
49  A debugging session is active.                      
50                                             
51          Inferior 1 [process 2873] will be killed.
52                                                                  
53  Quit anyway? (y or n) y     
\end{verbatim}
\end{document}